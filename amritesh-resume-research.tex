\documentclass[10pt, a4paper]{article}

% Packages:
\usepackage[
    ignoreheadfoot, % set margins without considering header and footer
    top=1 cm, % seperation between body and page edge from the top
    bottom=1 cm, % seperation between body and page edge from the bottom
    left=1 cm, % seperation between body and page edge from the left
    right=1cm, % seperation between body and page edge from the right
    footskip=1.0 cm, % seperation between body and footer
    % showframe % for debugging 
]{geometry} % for adjusting page geometry
\usepackage{titlesec} % for customizing section titles
\usepackage{tabularx} % for making tables with fixed width columns
\usepackage{array} % tabularx requires this
\usepackage[dvipsnames]{xcolor} % for coloring text
\definecolor{primaryColor}{RGB}{0, 0, 0} % define primary color
\usepackage{enumitem} % for customizing lists
\usepackage{fontawesome5} % for using icons
\usepackage{amsmath} % for math
\usepackage[
    pdftitle={Amritesh Amrit's CV},
    pdfauthor={Amritesh Amrit},
    pdfcreator={Amritesh Amrit},
    colorlinks=true,
    urlcolor=primaryColor
]{hyperref} % for links, metadata and bookmarks
\usepackage[pscoord]{eso-pic} % for floating text on the page
\usepackage{calc} % for calculating lengths
\usepackage{bookmark} % for bookmarks
\usepackage{lastpage} % for getting the total number of pages
\usepackage{changepage} % for one column entries (adjustwidth environment)
\usepackage{paracol} % for two and three column entries
\usepackage{ifthen} % for conditional statements
\usepackage{needspace} % for avoiding page brake right after the section title
\usepackage{iftex} % check if engine is pdflatex, xetex or luatex

% Ensure that generate pdf is machine readable/ATS parsable:
\ifPDFTeX
    \input{glyphtounicode}
    \pdfgentounicode=1
    \usepackage[T1]{fontenc}
    \usepackage[utf8]{inputenc}
    \usepackage{lmodern}
\fi

\usepackage{charter}

% Some settings:
\raggedright
\AtBeginEnvironment{adjustwidth}{\partopsep0pt} % remove space before adjustwidth environment
\pagestyle{empty} % no header or footer
\setcounter{secnumdepth}{0} % no section numbering
\setlength{\parindent}{0pt} % no indentation
\setlength{\topskip}{0pt} % no top skip
\setlength{\columnsep}{0.10cm} % set column seperation
\pagenumbering{gobble} % no page numbering

\titleformat{\section}{\needspace{4\baselineskip}\bfseries\large}{}{0pt}{}[\vspace{1pt}\titlerule]

\titlespacing{\section}{
    % left space:
    -1pt
}{
    % top space:
    0.1 cm
}{
    % bottom space:
    0.2 cm
} % section title spacing

\renewcommand\labelitemi{$\vcenter{\hbox{\small$\bullet$}}$} % custom bullet points
\newenvironment{highlights}{
    \begin{itemize}[
        topsep=0 cm,
        parsep=0 cm,
        partopsep=0pt,
        itemsep=0pt,
        leftmargin=0 cm + 10pt
    ]
}{
    \end{itemize}
} % new environment for highlights


\newenvironment{highlightsforbulletentries}{
    \begin{itemize}[
        topsep=0 cm,
        parsep=0 cm,
        partopsep=0pt,
        itemsep=0pt,
        leftmargin=10pt
    ]
}{
    \end{itemize}
} % new environment for highlights for bullet entries

\newenvironment{onecolentry}{
    \begin{adjustwidth}{
        0 cm + 0.00001 cm
    }{
        0 cm + 0.00001 cm
    }
}{
    \end{adjustwidth}
} % new environment for one column entries

\newenvironment{twocolentry}[2][]{
    \onecolentry
    \def\secondColumn{#2}
    \setcolumnwidth{\fill, 6.0 cm}
    \begin{paracol}{2}
}{
    \switchcolumn \raggedleft \secondColumn
    \end{paracol}
    \endonecolentry
} % new environment for two column entries

\newenvironment{threecolentry}[3][]{
    \onecolentry
    \def\thirdColumn{#3}
    \setcolumnwidth{, \fill, 4.5 cm}
    \begin{paracol}{3}
    {\raggedright #2} \switchcolumn
}{
    \switchcolumn \raggedleft \thirdColumn
    \end{paracol}
    \endonecolentry
} % new environment for three column entries

\newenvironment{header}{
    \setlength{\topsep}{0pt}\par\kern\topsep\centering\linespread{1.0}
}{
    \par\kern\topsep
} % new environment for the header

\newcommand{\placelastupdatedtext}{% \placetextbox{<horizontal pos>}{<vertical pos>}{<stuff>}
  \AddToShipoutPictureFG*{% Add <stuff> to current page foreground
    \put(
        \LenToUnit{\paperwidth-2 cm-0 cm+0.05cm},
        \LenToUnit{\paperheight-1.0 cm}
    ){\vtop{{\null}\makebox[0pt][c]{
        \small\color{gray}\textit{Last updated in September 2024}\hspace{\widthof{Last updated in September 2024}}
    }}}%
  }%
}%

% save the original href command in a new command:
\let\hrefWithoutArrow\href

% new command for external links:


\begin{document}
    \newcommand{\AND}{\unskip
        \cleaders\copy\ANDbox\hskip\wd\ANDbox
        \ignorespaces
    }
    \newsavebox\ANDbox
    \sbox\ANDbox{$|$}

    \begin{header}
    \fontsize{20 pt}{20 pt}\selectfont \textbf{Amritesh Amrit}

    \vspace{1pt} % Reduce space between name and tagline

    % \normalsize
    % Master’s in CS @ USC | 2 YOE at JPMorgan | PyTorch, AWS, Full-Stack | Applied AI \& Research Enthusiast

    % \vspace{2pt} % Reduce space between tagline and contact info

    \normalsize
    \href{mailto:aamrit@usc.edu}{aamrit@usc.edu} | \href{tel:+91-9108085826}{+91-9108085826} | \href{https://linkedin.com/in/amriteshamrit}{\underline{https://linkedin.com/in/amriteshamrit}} | \href{https://github.com/ammrit2312}{\underline{github.com/ammrit2312}}


%     \mbox{Los Angeles, CA}%
%      \textbar %
%     \href{mailto:amriteshc101@gmail.com}{amriteshc101@gmail.com}%
%     \quad\textbar\quad%
%     \href{tel:+1-541-999-9999}{+1-541-999-9999}%
%     \quad\textbar\quad%
%     \href{https://linkedin.com/in/amriteshamrit}{linkedin.com/in/amriteshamrit}%
%     \quad\textbar\quad%
%     \href{https://github.com/ammrit2312}{github.com/ammrit2312}
\end{header}

    \vspace{5 pt - 0.3 cm}

    \section{Education}
        \begin{twocolentry}{
            Aug 2025 - Present
        }
            \textbf{University of Southern California}, Los Angeles, CA, Master's in Computer Science\end{twocolentry}

        \vspace{0.10 cm}
        \begin{onecolentry}
            \begin{highlights}
                \item Incoming Student at the Viterbi School of Engineering (GRE Quants Score: 162)
                \item \textbf{Coursework:} Analysis of Algorithms
            \end{highlights}
        \end{onecolentry}

        \vspace{0.10 cm}
        \begin{twocolentry}{
            Aug 2019 - May 2023
        }
            \textbf{PES University}, Bengaluru, India, BTech. in Computer Science \& Engineering\end{twocolentry}

        \vspace{0.10 cm}
        \begin{onecolentry}
            \begin{highlights}
                \item GPA: 9.13/10 (Awarded Specialisation in Machine Intelligence \& Data Science)
                \item \textbf{Received Prof. MRD (6x, awarded to top 20\%) \& CNR Rao (1x, awarded to top 5\%) Scholarships}
            \end{highlights}
        \end{onecolentry}

    \section{Technologies \& Skills}
        \begin{onecolentry}
            \textbf{Languages:} JavaScript, Java, Python, HTML5, CSS, SQL, Swift 
        \end{onecolentry}
        \begin{onecolentry}
            \textbf{Frameworks:} ReactJS, Spring Boot, Node.js, Express, SwiftUI, UIKit, NextJS, Pytorch
        \end{onecolentry}
        \begin{onecolentry}
            \textbf{Tools:} AWS (Glue, EC2, ECS, S3, SQS, PySpark, Lambda), MongoDB, Git, MySQL, MS-SQL Server, PostgreSQL, XCode, Excel, Terraform, Github Actions
        \end{onecolentry}

    \section{Experience}
        \begin{twocolentry}{
            Jun 2023 – Jun 2025
        }
            \textbf{Software Engineer - I}, JP Morgan Chase \& Co. -- Bengaluru, India\end{twocolentry}
        \vspace{0.10 cm}
        \begin{onecolentry}
            \begin{highlights}
                \item Built a microfrontend architecture \& modern UI test suite; delivered scalable web-based tools using ReactJS \& Spring Boot with MS SQL \& ETL pipelines using AWS Glue, S3, \& PySpark.
                \item Delivered T+1 compliance dashboards (using ReactJS) to help global teams track trade failures across Fixed Income, Equities, Collateral, Commodities, \& other business lines.
                \item Optimized system performance by \textbf{reducing SQL job duration by ~75\% \& UI pipeline runtime by ~65\%}.
                \item Developed user-specific alert systems to monitor trade failures, \textbf{reducing the need for manual oversight by 30\%}.
                \item \textbf{Winner of the JP Morgan Chase Code for Good} hackathon 2021.
            \end{highlights}
        \end{onecolentry}


        \vspace{0.10 cm}

        \begin{twocolentry}{
            Feb – May 2023 \& Jun – Jul 2022
        }
            \textbf{Software Engineer Intern}, JP Morgan Chase \& Co. -- Bengaluru, India\end{twocolentry}

        \vspace{0.10 cm}
        \begin{onecolentry}
            \begin{highlights}
                \item Built UI to control \& monitor \textbf{6+ Kubernetes services}, enabling real-time toggling via APIs, log retrieval, \& pod-level access for improved system observability.
                \item Migrated 4 key Android screens from Java to Kotlin \& \textbf{improved readability \& maintainability across 80+ files}.
                \item Worked in Agile teams \& cross-functional teams to ship infrastructure tools \& scalable full-stack systems.
            \end{highlights}
        \end{onecolentry}

        \begin{twocolentry}{
            Jun – Dec 2021
        }
            \textbf{Research Intern}, CVIT, IIIT Hyderabad\end{twocolentry}

        \vspace{0.10 cm}
        \begin{onecolentry}
            \textit{Guide:} Ravi Kiran\\
            \begin{highlights}
                \item Engineered backend services for a real-time bot platform, seamlessly integrating a GAN model to generate user-customized images through an custom interactive ReactJS Canvas component.
                \item Developed a ReactJS frontend for an image-editing workflow, with the GAN model deployed via Flask APIs on a private server, enabling real-time interactive edits with instant visual feedback.
            \end{highlights}
        \end{onecolentry}
    
    \section{Publications}
        \begin{samepage}
            \begin{twocolentry}{
                \href{https://doi.org/10.1007/978-981-99-6984-5_15}{\underline{10.1007/978-981-99-6984-5\_15}}\\
            }
                \textbf{Localised Land-Use Classification Using U-Net and Satellite Imaging}
            \end{twocolentry}

            \vspace{0.10 cm}
            
            \begin{onecolentry}
                \begin{highlights}
                    \item Led a four-member team \& implemented a U-Net model with ResNet backbone for satellite image segmentation to classify land use types, including city, forest, roads (via QGIS), water, \& unused land.
                    \item Conducted ML experimentation, tuning, \& evaluation \textbf{achieving an IoU score of 0.68 against ground truth}.
                    \item \textbf{Awarded Best Paper Award} at the 2\textsuperscript{nd} International Conference on Intelligent Systems and Applications (ICISA 2023)
                \end{highlights}
        \end{onecolentry}
        \end{samepage}

    \section{Projects}
        
        % \begin{twocolentry}{
        %     Jan - Dec 2022
        % }
        %     \textbf{Localized Land-Use Classification Using U-Net and Satellite Imaging}\end{twocolentry}

        % \vspace{0.10 cm}
        % \begin{onecolentry}
        %     \textit{Guide:} S. Natarajan\\
        %     \begin{highlights}
        %         \item Led a four-member team \& implemented a U-Net model with ResNet backbone for satellite image segmentation to classify land use types, including city, forest, roads (via QGIS), water, \& unused land.
        %         \item Conducted ML experimentation, tuning, \& evaluation achieving an IoU score of 0.68 against ground truth.
        %     \end{highlights}
        % \end{onecolentry}

        \begin{twocolentry}{}
            \textbf{Pothole Detection – YOLOv5}\end{twocolentry}

        \vspace{0.10 cm}
        \begin{onecolentry}
            \begin{highlights}
                \item Engineered a pothole detection model using YOLOv5, \textbf{enhancing the accuracy by 27\%} allowing user complaints to be submitted quickly using the chatbot.
                \item Collaborated with cross-functional teams to deliver this project, \textbf{accelerating the project timeline by 30\%}.
            \end{highlights}
        \end{onecolentry}
        
        \vspace{0.10 cm}

        \begin{twocolentry}{
            \href{https://github.com/ammrit2312/treasure-server}{\underline{Github}}
        }
            \textbf{Treasure (Startup)} - Co-founder \& Lead backend developer\end{twocolentry}

        \vspace{0.10 cm}
        \begin{onecolentry}
            \begin{highlights}
                \item Raised \$200K in seed funding for a platform enabling micro \& nano influencers to build brand partnerships beyond barter-based models.
                \item Designed RESTful APIs using Node.js microservices with MySQL, S3, SQS; added NFT generation, cron jobs, \& user auth. 
            \end{highlights}
        \end{onecolentry}

        \vspace{0.10 cm}
        \begin{twocolentry}{
            \href{https://www.whywaste.io/}{\underline{Why Waste?}}
        }
            \textbf{Why Waste? Website}
        \end{twocolentry}

        \vspace{0.10 cm}
        \begin{onecolentry}
            \begin{highlights}
                \item Designed and implemented the main website for Why Waste? moving it from wordpress to ReactJS, \textbf{reducing the development time by 40\%}. 
                \item Deployed the RESTful API \& UI on Vercel using Github Actions, \textbf{lowering the operational cost by 38\%}.
            \end{highlights}
        \end{onecolentry}


        \vspace{0.10 cm}

        % \begin{twocolentry}{
        %     \href{https://github.com/ammrit2312/Merobot/tree/master}{Github (Jan - Dec 2021)}
        % }
        %     \textbf{Merobot – GAN-based Bot System} - Research Intern, CVIT, IIIT Hyderabad\end{twocolentry}

        % \vspace{0.10 cm}
        % \begin{onecolentry}
        %     \textit{Guide:} Ravi Kiran\\
        %     \begin{highlights}
        %         \item Developed backend services for a real-time bot system that interacted with a GAN model to generate images based on user-provided inputs.
        %         \item Enabled image-editing workflows in which each user modification triggers new GAN outputs, supporting interactive visual feedback.
        %     \end{highlights}
        % \end{onecolentry}

    \section{Certifications}

    \begin{onecolentry}
        \textbf{AWS Certified Cloud Practitioner}, Amazon Web Services\hfill \href{https://cp.certmetrics.com/amazon/en/public/verify/credential/21NY0Y6C714E1JSB}{\underline{Credential Link}}
    \end{onecolentry}
    
    \begin{onecolentry}
        \textbf{30 Days of Google Cloud Program}, Google\hfill \href{https://drive.google.com/file/d/131IwLvWr8a6YWLYd5ILZ6kvacsKfmkWc/view}{\underline{Credential Link}}
    \end{onecolentry}
    

\end{document}